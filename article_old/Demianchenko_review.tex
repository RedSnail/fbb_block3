\documentclass{article}
\usepackage[utf8]{inputenc}
\usepackage[russian]{babel}

\title{Обзор на бактерию Helicobacter cinaedi CCUG 18818 = ATCC BAA-847}
\date{17 ноября 2019}
\author{Демьянченко Олег}

\begin{document}
\maketitle
\section{Введение}
Helicobacter относятся к типу протеобактерии (в нём-же находятся E. coli и Neisseria)
Различные представители Helicobacter живут кишечнике и печени млекопитающих и птиц.
Helicobacter олигоаэробны, хемоорганогетеротрофны, способны к нитратредукции.\newline
Данные, представленые ниже взяты из статьи (Yoshiaki Kawamura, 2014)[1]
Конкретно Helicobacter cinaedi был выделен из людей и хомяков, где он обычно живёт в печени и кишечнике.
Ранее считалось, что заболевание H. cinaedi может вызывать только у людей с иммунодеффицитом, однако позже были выявлены случаи заражения имунокомпетентных людей.
При заражении человека бактерия выходит в кровь, что вызывает повышение температуры и воспаление.
На мышах было показано, что H. cinadei серьёзно увеличивает риск развития атеросклероза.
Также этот вид часто обнаруживался в бляшках людей, умерших от атеросклероза, что делает эту бактерию интересной для исследования. 
\section{Материалы и методы}
\begin{itemize}
\item{работа с excel}
\begin{itemize}
\item{импорт tsv файлов}
\item{фильтрация}
\item{относительная и абсолютная адресация}
\item{специальная ставка}
\item{функция ВПР (также известная как VLOOKUP)}
\item{условное форматирование}
\end{itemize}
\end{itemize}

\section{Источники}
[1] - https://www.sciencedirect.com/science/article/pii/S1341321X14002256
\end{document}

